\documentclass[platex,a4paper,12pt,dvipdfmx]{jsarticle}
\usepackage{graphicx}
\usepackage{float}
\usepackage{tabularx}
\usepackage{wrapfig}

\begin{document}
\section{図の例}
\begin{figure}[hbt]
  \begin{center}
    % scale,heightなどでも大きさを制御できます。
    \includegraphics[width=24em]{/home/course/literacy/pub/sample/FIGURE/CentOS_Firefox.png}
  \end{center}
  \caption{\texttt{\textbackslash{}includegraphics}命令と\texttt{figure}環境を用いた図版の例}
  \label{ex-latex:subwindow}
\end{figure}

画像の一部を切り出して表示することもできます。

\begin{figure}[hbt]
  \begin{center}
    \includegraphics[width=24em,trim=0 720 360 72,clip]
    {/home/course/literacy/pub/sample/FIGURE/CentOS_Firefox.png}
  \end{center}
  \caption{タイトル、キャプション}
  \label{ex-latex:subsubwindow}
\end{figure}

\subsection{図番号の例}
図\ref{ex-latex:subwindow} はFirefoxの起動画面です. 図\ref{ex-latex:subsubwindow} は図\ref{ex-latex:subwindow} の一部を切り出したものです.

\section{表の例}
\subsection{tabular環境を利用した表の例}

\subsubsection{シンプルな表の例}

\begin{tabular}{|r|c|l|p{5em}|}
  \hline
  Column1    & Column2 & Column3 & Column4\\
  \hline
   Right    & Left & Center &  MMMMM (5 em)\\
  \hline
  Aaaaaaa     & Bbbbbbb & Cccccc & Ddddddd\\
  \hline
\end{tabular}

\subsubsection{横罫線のみのシンプルな表の例}

\begin{tabular}{rclp{5em}}
  \hline
  Column1    & Column2 & Column3 & Column4\\
  \hline
   Right    & Left & Center &  MMMMM\\
  Dd     & E & F & G\\
  \hline
\end{tabular}


\subsubsection{複雑な表の例}
\begin{tabular}[t]{|l|c|c|} %3つのカラムごとの文字揃え
  \hline
 \multicolumn{3}{|c|}{専門基礎科目一覧} \\
  \hline
  確率統計学    & 2 & 3年前期 \\ % & でカラムの区切りを, \\ で行の終わりを示す
  幾何学I       & 2 & 2年後期 \\
  幾何学II      & 2 & 3年前期 \\
  量子力学      & 2 & 2年後期 \\
  統計力学      & 2 & 2年後期 \\
  \hline
  単位合計      & 12 \\
  \cline{1-2}
\end{tabular}

\subsection{tabularx環境を利用した表の例}
表\ref{tableexample}はtabularx環境を用いて作成されているとともに、table環境の中に入れて図版のfigure環境と同様、ページ中の配置の制御とcaption命令による表タイトルの付加も行なっている。table環境はtabular環境で作成した表にも適用できる。

\begin{table}[htb]
  \caption{表のタイトル(一般的には表の上に書く)}
  \begin{tabularx}{33zw}{| c || X |  X | X |} %ハンドアウトの例から横幅を調整している
    \hline
    番号 &  料理 & 材料 (4皿分) & 一皿当りのカロリー (Kcal)\\
    \hline
    \hline
    1 &  カレー & 肉 200g、玉ねぎ2、人参2、じゃがいも2、市販のルーなど & 900 \\
    \hline
    2 &  シチュー& 肉 200g、玉ねぎ2、人参2、じゃがいも2、市販のルーなど & 900 \\
    \hline
    3 &  肉じゃが & 肉 200g、玉ねぎ2、人参2、じゃがいも2、糸こんにゃく200g、醤油、砂糖 & 900 \\
    \hline
  \end{tabularx}
  \label{tableexample}
\end{table}

\newpage

\section{図表と文章のレイアウトの例}

{\LaTeX}文書の組版の上では、\texttt{figure} 環境や\texttt{table} 環境を使わない図表は大きな「文字」として扱われます。このため、それらの図表はこの画像 \includegraphics[width=4zw]{/home/course/literacy/pub/sample/FIGURE/uoa_logo.png} のように本文中に入り込む形でレイアウトされます。

\begin{wrapfigure}{r}{10zw}
\includegraphics[width=10zw]{/home/course/literacy/pub/sample/FIGURE/uoa_logo.png}
\caption{\texttt{wrapfigure} 環境を用いた図の配置の例}\label{ex:latex_wrapfigure}
\end{wrapfigure}

\texttt{wrapfig}パッケージに含まれている \texttt{wrapfigure} 環境を使うと、図の周りに本文を回り込ませるレイアウトが実現できます。図\ref{ex:latex_wrapfigure}が\texttt{wrapfigure} 環境を用いた実例です。\verb+\begin{wrapfigure}{r}{10zw}+の\verb+{r}+でページ中に右寄せで図が配置されます。左寄せにする場合は\verb+{l}+とします。\verb+{10zw}+は図の幅を指定しています。表を回り込み配置したい場合は同じく \texttt{wrapfig}パッケージに含まれている\texttt{wraptable} 環境を使うことができます。



\section{マクロの例}
\newcommand{\uoa}{会津大学コンピュータ理工学部}
\newcommand{\midasi}[1]{\vspace{0.5cm}\par\noindent%
\rule[2pt]{\textwidth}{1pt}\\{\large\gt\bf{#1}}\\%
\rule[8pt]{\textwidth}{2pt}\par\vspace{0.5cm}}

以下の文章は\texttt{{\textbackslash}uoa}マクロを使って出力している。

ようこそ{\uoa}へ

\midasi{midasiマクロを使って見出しを作成}

\texttt{{\textbackslash}midasi}マクロはフォントサイズを変え、上下に線を引いた見出し行を作り出すマクロです。マクロの定義内容のうち一部を書き換えて(例:
\texttt{{\textbackslash}rule}の数値を変更する、字体の指定を変更するなど)、再度コンパイルしてみましょう。出力結果のどこがどの様に変化するか、観察するとマクロの仕組みの一端を理解できるでしょう。

\end{document}
